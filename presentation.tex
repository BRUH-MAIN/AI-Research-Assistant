\documentclass{beamer}
\usepackage{ctex, hyperref}
\usepackage[T1]{fontenc}

% other packages
\usepackage{latexsym,amsmath,xcolor,multicol,booktabs,calligra}
\usepackage{graphicx,pstricks,listings,stackengine,hyperref}

\title{AI Research Assistant}
\subtitle{A review}
\author{\href{}{Team - 08}}
\institute{\href{}{School of AI, Amrita Vishwa Vidyapeetham}}

\date{}
\usepackage{Amr_Beamer}

% defs
\def\cmd#1{\texttt{\color{red}\footnotesize $\backslash$#1}}
\def\env#1{\texttt{\color{blue}\footnotesize #1}}
\definecolor{deepblue}{rgb}{0,0,0.5}
\definecolor{deepred}{rgb}{0.6,0,0}
\definecolor{deepgreen}{rgb}{0,0.5,0}
\definecolor{halfgray}{gray}{0.55}

\lstset{
    basicstyle=\ttfamily\small,
    keywordstyle=\bfseries\color{deepblue},
    emphstyle=\ttfamily\color{deepred},   % Custom highlighting style
    stringstyle=\color{deepgreen},
    numbers=left,
    numberstyle=\small\color{halfgray},
    rulesepcolor=\color{red!20!green!20!blue!20},
    frame=shadowbox,
}

\begin{document}

\begin{frame}
    \titlepage
    \begin{figure}
            \includegraphics[scale=0.2]{pic/logored.png} 
    \end{figure}
\end{frame}

\begin{frame}
    \tableofcontents[sectionstyle=show,subsectionstyle=show/shaded/hide,subsubsectionstyle=show/shaded/hide]
\end{frame}


% 内容从这里开始
\section{Introduction}
\begin{frame}{Introduction}
    \begin{itemize}
  \item Researchers struggle to efficiently find and use relevant papers due to the vast volume of academic content.
  \item Manual searching, reading, and summarizing papers slows down research and reduces productivity.
  \item Technical terms and complex concepts in papers create barriers, especially for students and new researchers.
  \item Properly citing sources and adhering to formatting guidelines is error-prone and time-intensive.
  \item  Academia and industry increasingly need AI tools to simplify research, improve accuracy, and boost efficiency.
\end{itemize}
\end{frame}

\section{Problem Statement}
\begin{frame}{Problem Statement}
    \begin{itemize}
  \item Slow, error-prone, and time-consuming manual processes hinder productivity in searching, analyzing, and citing research papers.
  \item Such problems can be overcome by integrating AI services with the research process. 
  \item A high-performance database is critical to manage growing data, ensure fast queries, and support real-time AI processing for seamless user experiences.
\end{itemize}
\end{frame}

\section{Objective}
\begin{frame}{Objective}
   \begin{itemize}
  \item Develop an AI-powered research platform with capabilities for group collaboration, role-based access for mentors and researchers, and intelligent paper management.
  \item Design a normalized relational database schema in PostgreSQL to store and manage user profiles, mentor-researcher relationships, research papers, citations, and collaborative projects, ensuring data consistency and optimized CRUD operations.
  \item Integrate the arXiv API with a Retrieval-Augmented Generation (RAG) engine to enable advanced paper exploration, conceptual querying, and automated citation preparation.
\end{itemize}
\end{frame}

\section{Tech Stack}
\begin{frame}{Tech Stack}
    % Content for Tech Stack
    \begin{itemize}
    \item Frontend: Next.js with TypeScript
    \item Backend: FastAPI
    \item Database: PostgreSQL
    \item UI/Styling: Aceternity UI + Tailwind CSS
    \item Caching: Redis for high-performance caching
    \end{itemize}
\end{frame}

\section{ER Model}
\begin{frame}{ER Model}
    % Content for ER Model
    \begin{figure}
        \centering
        \includegraphics[width=0.95\linewidth]{ER-Final[1].png}
        \caption{ER Model}
        \label{fig:placeholder}
    \end{figure}
\end{frame}

\section{UML Class Diagram}
\begin{frame}{UML Class Diagram}
    % Thumbnail
    \includegraphics[height=0.5\linewidth]{uml_class.pdf}
    \vspace{1em}

    % Button linking to the Drive-hosted PDF
    \href{https://drive.google.com/file/d/1zFV6mQx6Ovp5fQPJ1y4IFWDJutJeNi6X/view?usp=drive_link}{\beamergotobutton{Open UML Class Diagram}}
\end{frame}

\section{UML Sequence Diagram}
\begin{frame}{UML Sequence Diagram}
    \centering
    % Display the SVG
    \includegraphics[height=0.5\linewidth]{seq2.png}

    \vspace{1em}

    % Optional: Keep the Drive link button if you still want it
    \href{https://drive.google.com/file/d/1nDxdsARpn7qJ2UEqsf4UgOmcG2kOt_gX/view?usp=sharing}{\beamergotobutton{Open Full Diagram}}
\end{frame}

\end{document}